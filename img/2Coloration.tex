\documentclass{standalone}

\usepackage{TikzStyle}
\usepackage{mystyle}

\def\dist{1}

\begin{document}
    \begin{tikzpicture}
    [   node distance=\dist cm,on grid,auto,
        every node/.style= {circle,draw},
    ]
    \node (n1) at (0,0) {};
    \node (n2) [right=of n1] {};
    \node (n3) [above right=of n2] {};
    \node (n4) [below right=of n2] {};
    \draw (n1) -- (n2) -- (n3) -- (n2) -- (n4);
    \draw [->] (0.3,4pt) -- (0.6,4pt);

    \draw [->] (2,0) -- (2.5,0);

    \node (n11) at (3,0) {};
    \node[fill=black] (n12) [right=of n11] {};
    \node (n13) [above right=of n12] {};
    \node (n14) [below right=of n12] {};
    \draw (n11) -- (n12) -- (n13) -- (n12) -- (n14);
    \draw [->] (4.1,0.3) -- (4.4,0.6);
    \draw [->] (4.1,-0.3) -- (4.4,-0.6);

    \draw [->] (5,0) -- (5.5,0);

    \node (n21) at (6,0) {};
    \node[fill=black] (n22) [right=of n21] {};
    \node (n23) [above right=of n22] {};
    \node (n24) [below right=of n22] {};
    \draw (n21) -- (n22) -- (n23) -- (n22) -- (n24);
    \node[draw=white] (done) at (8.5,0) {\textsc{done}}; % Trucco lercio per non vedere il bordo, fa
    % schifo
    \end{tikzpicture}
\end{document}
