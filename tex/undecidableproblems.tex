\chapter{Problemi indecidibili}

Ci chiediamo ora se ci sono dei problemi che non sono decidibili, ovvero non calcolabili in un
formalismo Turing completo. Introduciamo l'argomento con uno dei problemi più noti in letteratura,
l'\textit{Halting problem}.

Prima però chiariamo meglio i concetti di totalità, calcolabilità e la relazione tra i due, oltre
a introdurre i concetti di divergenza e convergenza.

\section{Note introduttive su programmi e funzioni}

Una qualsiasi funzione $\phii$ della nostra enumerazione delle funzioni calcolabili è una funzione
parziale calcolabile: su alcuni input può non essere definita. 

La calcolabilità non coincide con la totalità: esistono funzioni parziali calcolabili e funzioni
totali non calcolabili.

Le funzioni possono essere non definite in un punto. Se questo è il caso per una funzione
calcolabile $i$ ed un punto $x$, avremo allora la divergenza del \emph{programma} $i$: $\phii(x)
\diverges \iff \phii$ non è definita su input $x$. Divergenza e convergenza sono più proprietà del
programma che della funzione che calcola. Nonostante qui usiamo il simbolo $\phi$ sia per i
programmi che per le funzioni è bene ricordare che può avere un doppio ruolo e una semantica
diversa associata ad esso. La relazione tra i due è: la funzione $i$ nella mia enumerazione di
funzioni calcolabili è quella calcolata dal programma $i$ della mia enumerazione dei programmi.

La divergenza non è una proprietà generale delle funzioni. Non ha senso dire $\phii \diverges$ senza
specificare dove diverge; questo perché la convergenza e la divergenza sono proprietà puntuali:
valgono per un dato input. Al massimo $\phii(x)$ per un qualche $x$ può divergere.

\section{Il problema della fermata}

Il problema che ci interessa è il cosiddetto ``problema della terminazione''. Le funzioni che
stiamo calcolando sono funzioni parziali: i programmi corrispondenti possono potenzialmente
divergere. Noi vorremmo calcolare la seguente funzione:
\begin{equation*}
\textit{Term}(i,x) =
    \begin{cases}
        1 \quad & \text{se $\phii(x) \converges$} \\
        0 \quad & \text{se $\phii(x) \diverges$} 
    \end{cases}
\end{equation*}

Questà è la specifica della mia funzione. È una funzione totale. Ci chiediamo a questo punto, è
anche calcolabile? La risposta, come vedremo, sarà negativa.

Per dimostrarlo supponiamo che \textit{Term} sia calcolabile e prendiamo in considerazione ora la
funzione intermedia $g$:
\begin{equation*}
    g(x) =
    \begin{cases}
        1 \quad & \text{se $\textit{Term}(x,x) = 0$} \\
        \diverges \quad & \text{se $\textit{Term}(x,x) = 1$} 
    \end{cases}
\end{equation*}

Se \textit{Term} fosse calcolabile allora anche $g$ sarebbe calcolabile. Se $g$ è calcolabile deve
esistere un $k$ tale che $\phik = g$. Ci può essere più di un programma che calcola g, ma a me ne
basta uno.

Ci chiediamo ora, legittimamente, qual è il comportamento di $\phik(k)$? Converge?

Abbiamo che $\phik(k) = g(k)$. Quindi $\phik(k) \diverges \iff \textit{Term}(k,k) = 0 \iff g(k) = 1
\iff g(k) \converges \iff \phik(k) \converges$. Questo è contraddittorio. Verrebbe da concludere che
$\phik(k)$ converge. È tuttavia vero che $\phik(k) \converges \iff \textit{Term}(k,k) = 1 \iff g(k)
\diverges \iff \phik(k) \diverges$. Ma anche questo è contradditorio. L'ipotesi da cui siamo partiti
è che \textit{Term} fosse calcolabile. Concludiamo quindi che \textit{Term} non è calcolabile.

%T(i,x) = 
%\begin{cases}
%    1 se \phii(x) è definita \\
%    0 se \phii(x) è indefinita
%\end{cases}
%
%Useremo la notazione \phii(x) \converges e \phii(x) \diverges.
%
%Definiamo allora g(x):
%
%g(x) =  se T(x,x) = 0 restituisco 1
%        else \diverges
%
%Supponiamo che g sia \phi_{k}. Ci chiediamo: quanto vale T(k,k)? Ci sono due possibilità:
%
%0 = T(k,k) = g(k) = 1
%
%che è assurdo, oppure:
%
%1 = T(k,k) = g(k) = \diverges
%
%che è anche assurdo. Questo implica l'assurdità e la non calcolabilità di T.

È il primo caso di una funzione che possiamo pensare ma che non riusciamo a calcolare, almeno con questa
formulazione qui.

La dimostrazione è un semplice ragionamento diagonale. Esistono quindi funzioni ben definite ma non
calcolabili: non esiste un algoritmo che mi calcoli questo problema. Nella sua forma generale il
problema delle terminazione non è algoritmico.

Cosa vuol dire nella sua forma generale? Significa che valga per tutti gli $i$ e per tutti gli $x$.
Per alcuni programmi e alcuni input è possibile dimostrare che il programma termina. Ci sono dei
casi particolari che sono gestibili, ma non esiste un unico algoritmo che in modo uniforme su tutti
gli $i$ e tutti gli $x$ sappia decidere se il programma $i$ termini su input $x$.

Quali erano le nostre ipotesi? La calcolabilità dell'interprete e qualche piccola proprietà di
chiusura sul mio formalismo. Non molto.

Questa funzione non è esprimibile in un formalismo Turing completo. Non è tuttavia assurda l'idea
che esista un agente di calcolo con più capacità della macchina di Turing e che sia in grado di
calcolare \textit{Term}. È difficile da immaginare. Nella calcolabilità relativa si parte
immaginando un oracolo che sia capace di calcolare \textit{Term} e ci si chiede da lì cosa si possa
calcolare (e cosa no).

\section{Proprietà s-m-n}

%// TODO Va tenuta questa parte? C'entra poco.
%Nota: Quando ho statements del tipo \forall x \exists y, dove ho dipendenze funzionali tra y e x, se ho che
%ogni y è calcolabile in funzione di ogni x ho un teorema/dimostrazione intuizionistica/costruttiva.
%Altrimenti siamo nella matematica classica (dove vale il princio del terzo escluso e la RAA e dove
%nella mia dimostrazione non mostro gli x e y per cui vale). 

L'ipotesi della calcolabilità dell'interprete è un'ipotesi importante. Supponiamo, nella nostra
teoria della calcolabilità, che esista un modo per calcolarla. Questa proprietà è detta proprietà
UTM, Universal Turing Machine: $\exists u, \phiu(i,x) = \phii(x)$. È dimostrabile in tutti i
formalismi Turing completi.

La proprietà che andiamo ora a considerare è la cosiddetta proprietà s-m-n. Supponiamo di avere
una funzione calcolabile $g(x,y)$. Cosa posso dire delle sue istanze? Supponiamo di fissare $x$, ad
esempio a 0. Ottengo $g(0,y)$, che è una funzione unaria che dipende solo da $y$. Possiamo indicare
$g(0,y)$ come $f_{0}(y)$. In generale posso fare questo per $f_{k}(y) = g(k,y)$.

Se tutte queste funzioni sono calcolabili per ognuna c'è una qualche funzione, delle mia
enumerazione, che la calcola. Esisterà quindi un programma che la calcola. Esiste quindi
$\phi_{h(k)}(y) = f_{k}(y) = g(k,y)$ che mi calcola $g(k,y)$. $h$ è una funzione che mi calcola
l'indice del programma che calcola la $g$ che mi interessa. L'esistenza di $h$ è praticamente ovvia:
se esiste un indice $k'$ per la funzione $g(k,y)$ allora questo $k'$ è calcolabile (mediante $h$).
La domanda ora è: $h$ è calcolabile? La risposta è sì.

Noi supporremo la proprietà s-m-n, ma questa è facilmente dimostrabile in tanti formalismi.
\begin{thm}
    \textbf{Proprietà s-m-n. } $\forall g$ calcolabile $\exists h$ totale e calcolabile tale che
    \begin{equation*}
        \forall x,y. \phi_{h(x)}(y) = g(x,y)
    \end{equation*}
\end{thm}

Quello che stiamo facendo è una curryficazione. C'è un importante isomorfismo a livello di funzioni:
lo spazio delle funzioni $(A \times B) \to C$ è isomorfo allo spazio delle funzioni $A \to (B \to
C)$. L'operazione alla base della dimostrazione di questa affermazione e che mi permette di passare
da una funzione del primo spazio alla sua corrispondente nel secondo si chiama curryficazione.
L'idea è: data $g(x,y)$ fissiamo $x$. In questo modo ottengo $g(x,y)$, con $x$ fissato, ovvero una funzione
unaria in $y$ da $B$ a $C$.

Si può dimostrare anche con un argomento sulla cardinalità. Sappiamo che la cardinalità del primo
spazio è $|C|^{|A \times B|} = |C|^{|A|\cdot|B|} = (|C|^{|B|})^{|A|}$, che è la cardinalita di $A
\to B \to C$. Questo mi garantisce l'esistenza dell'isomorfismo. La dimostrazione precedente è un pò
più strutturale (e costruttiva se così si vuol dire).

$h$ fondamentalmente mi dà l'indice della funzione curryficata.

In un certo senso s-m-n mi dice che il mio formalismo è chiuso rispetto alle
curryficazioni/$\lambda$-astrazioni. UTM mi dice che il mio formalismo è chiuso rispetto alle
$\lambda$-applcazioni.

Possiamo generalizzare s-m-n a vettori $x$ e $y$ di parametri: 
\begin{thm}
    $\forall g \forall m \forall n \exists s$ totale calcolabile tale che
    \begin{equation*}
        \phi_{s(\vec{x}_{m})}(\vec{y}_{n}) = g(\vec{x}_{m},\vec{y}_{n})
    \end{equation*}
\end{thm}

s-m-n serve per calcolare un numero come indice di un programma. Se non bisogna calcolare un indice
di un programma in funzione di qualcos'altro non mi serve s-m-n. Noi vedremo molti casi in cui
avremo proprio bisogno di quello.

Vediamo un esempio di applicazione di s-m-n nella dimostrazione della non calcolabilità di alcune
funzioni.

La funzione che ci interessa indagare adesso è \textit{Tot} che determina se un certo programma è
totale o meno. È anche questo un problema di decisione. La specifica della mia funzione è:
\begin{equation*}
    \textit{Tot}(i) =
    \begin{cases}
        1 \quad &\text{se $\forall x \phii(x) \converges$} \\
        0 \quad &\text{altrimenti}
    \end{cases}
\end{equation*}

Abbiamo bisogno però prima di un risultato intermedio. Un caso particolare caso del problema della
terminazione è il problema della terminazione diagonale:
\begin{equation*}
    \textit{Term}(i) =
    \begin{cases}
        1 \quad &\text{se $\phii(i) \converges$} \\
        0 \quad &\text{se $\phii(i) \diverges$} \\
    \end{cases}
\end{equation*}

Se la terminazione diagonale non è calcolabile la terminazione generabile non è calolabile. Non è
vera l'implicazione inversa. Ha senso quindi chiedersi se la terminazione diagonale è calcolabile.
La risposta è no, e la dimostrazione è analoga a quella della terminazione generale.

Dimostrare che una funzione non sia calcolabile non è banale. Devo dimostrare che non esiste un
algoritmo tale che la calcoli. È molto più difficile rispetto a dimostrare la calcolabilità di una
funzione.

Abbiamo due strade. Assumiamo la calcolabilità di \textit{Tot} e troviamo un assurdo. Oppure usiamo
un procedimento di riduzione: se è calcolabile \textit{Tot} deve essere calcolabile
$\textit{Term}_{i}$. Da lì poi dimostriamo la non calcolabilità di $\textit{Term}_{i}$.

Prendiamo $g(i,x) = \phii(i)$. Abbiamo due casi: o converge o diverge. Nel primo caso se fisso $i$
la funzione curryficata che ottengo è totale. Nell'altro caso no.

Per s-m-n abbiamo h totale e calcolabile tale che
\begin{equation*}
    \phi_{h(i)}(x) = g(i,x) = \phii(i)
\end{equation*}

Ora mi chiedo: quanto vale $\textit{Tot}(h(i))$? Abbiamo che
\begin{equation*}
    \textit{Tot}(h(i))=
    \begin{cases}
        1 \quad &\text{se $\phii(i) \converges$} \\
        0 \quad &\text{se $\phii(i) \diverges$}
    \end{cases}
\end{equation*}

Componendo \textit{Tot} con $h$ risolverei il problema della terminazione diagonale. Ma questo è
assurdo. Da cui \textit{Tot} non è calcolabile.

In questa dimostrazione parto da funzioni calcolabili che vado a comporre con la mia funzione di
partenza (\textit{Tot}) e ottengo una funzione che mi potrebbe calcolare qualcosa che so non essere
calcolabile. Da ciò concludo che la mia funzione di partenza non è calcolabile. È una
dimostrazione diversa da quella del problema della terminazione.

È interessante vedere questa dimostrazione anche ``al contrario'', in versione bottom up. Come
faccio a dimostrare la non calcolabilità di \textit{Tot}? Assumo che sia calcolabile e la uso per
risolvere un problema indecidibile. In questo caso vogliamo ridurci alla terminazione diagonale.
Come facciamo? Possiamo farlo cercando una funzione $h$ tale che, per ogni $i$, $\phi_{h(i)}$ è totale sse
$\phii(i) \converges$. Esiste questa funzione $h$? Sì, e possiamo dimostrarlo in due passaggi.

Per prima cosa definiamo una funzione calcolabile binaria $g(i,x)$ che, in base alla terminazione di
o meno di $\phii(i)$, ha un comportamento diverso che rispetti le condizioni che abbiamo posto su
$\phi_{h(i)}$. Ad esempio
\begin{equation*}
    g(i,x) =
    \begin{cases}
        1 \quad &\text{se $\phii(i)\converges$} \\
        \diverges \quad &\text{se $\phii(i)\diverges$} \\
    \end{cases}
\end{equation*}

Questa funzione è calcolabile? Sì, è facilmente scrivibile in un linguaggio di programmazione. Ma a
questo punto ho trovato la $h$ che cercavo, grazie alla proprietà s-m-n: $\exists h \text{ tot. e
calc.} \phi_{h(i)}(x) = g(i,x)$. A questo punto, se mi chiedo se la funzione $\phi_{h(i)}$ è totale,
utilizzando la funzione \textit{Tot}, mi sono ridotto al problema della terminazione diagonale.

Consideriamo un caso particolare e poi proviamo a generalizzare. Stiamo analizzando programmi. Mi
chiedo se un mio programma calcola la funzione identità. Si potrebbe generalizzare al capire se il
mio programma ha un certo comportamento rispetto ad una funzione di riferimento.

Vogliamo quindi
\begin{equation*}
    \textit{ID}(i) =
    \begin{cases}
        1 \quad &\text{se $\forall x, \phii(x) = x$} \\
        0 \quad &\text{altrimenti}
    \end{cases}
\end{equation*}

Possiamo dimostrare che questa funzione non è calcolabile, con la stessa tecnica di prima. Ci
riduciamo al problema della terminazione.

Costruiamo $g$ tale che
\begin{equation*}
    g(i,x) =
    \begin{cases}
        x \quad & \text{se $\phii(i) \converges$} \\
        \diverges \quad & \text{se $\phii(i) \diverges$}
    \end{cases}
\end{equation*}

Per s-m-n esiste $h$ tale che $\phi_{h(i)}(x) = g(i,x)$. Mi chiedo ora, $h(i)$ è la funzione
identita?

\begin{equation*}
    \textit{ID}(h(i)) =
    \begin{cases}
        1 \quad &\text{se $\phii(i) \converges$} \\ 
        0 \quad &\text{ se $\phii(i) \diverges$} \\
    \end{cases}
\end{equation*}

Bisogna partire da una funzione calcolabile, altrimenti non si può applicare s-m-n. È parte
fondamentale della dimostrazione mostrare che $g$ è calcolabile.

\section{Il predicato $T$ di Kleene}

Ci chiediamo ora se esista un predicato calcolabile $T(i,x,t)$ che mi dice se il programma $i$
termina la computazione su input $x$ entro il tempo $t$.

C'è il problema di come definiamo il tempo; tuttavia l'idea è che tutti i programmi di cui
parliamo sono di tipo discreto, hanno un passo discreto di calcolo.

È calcolabile? Intuitivamente sì. Immaginiamo di avere un interprete. Interpretiamo il programma
$i$, seguendolo passo per passo per input $x$. Se dopo $t$ passi la computazione è terminata allora
ho una risposta positiva; in caso contrario no.

Esistono tante varianti del predicato $T$ di Kleene. Questa è una forma ``ternaria''. Possiamo pensare
ad una versione ``quaternaria'' $T^{4}(i,x,y,t)$ che mi dice se il mio programma termina su output
$y$ in $t$ o meno passi su input $x$.

La forma originale di Kleene era un predicato ternario $T(i,x,\textit{tr})$, dove \textit{tr} è una
traccia computazionale. Si può vedere come una sequenza di configurazioni istantanee. Non deve
essere necessariamente completa, deve essere corretta. È ancora chiaramente decidibile se la traccia
seguita sia quella passata in input. Questa forma in un certo senso comprende le prime due. L'idea è
che la lunghezza della traccia è il tempo passato dall'inizio della computazione.

Si può addirittura dimostrare che $T$ è primitivo ricorsivo in generale. Inoltre ha una complessità
computazionale relativamente bassa. La complessità è lineare in $t$ e nelle altre componenti.

C'è un corollario del predicato $T$ di Kleene. Se supponiamo questo predicato come primitivo
possiamo scrivere, sulla base di questo, la macchina universale, la cui esistenza smette di essere
primitiva nella nostra teoria della calcolabilità.

Infatti possiamo definire $u$ nel seguente modo:

\begin{equation*}
    u(i,x) = \textit{fst}(\mu <y,t>, T(i,x,y,t))
\end{equation*}

Questo corrisponde a scrivere $\textit{fst}(\mu w, T(i,x,\textit{fst}(w),\textit{snd}(w)))$. Il
\textit{fst} più esterno serve perché a me non interessa $t$, interessa $y$.

Questo è un risultato importante e si chiama forma normale di Kleene. C'è un corollario
importante. Si potrebbe limitare sintatticamente un programma ad un while con all'interno un for e
questo non diminuirebbe il potere espressivo del formalismo, poichè l'interprete è scrivibile in
questi termini.

Un altro problema non calcolabile è il problema del raggiungimento di codice. Ovvero, dato un
programma e un'istruzione il problema di decidere se il programma raggiungerà mai quell'istruzione
non è calcolabile. Esistono delle tecniche ma queste non sono generali. Sono tipicamente usate dai
compilatori per ottimizzare il codice oggetto ed eliminare parti di codice inutile.
