\chapter{Esercizi svolti in classe}

\section{Compito Parziale Gennaio 2018}

\subsection{Esercizio 1}

Data una funzione $f:\Nat \to \Nat$ e un insieme $A \subseteq N$, la controimmagine di $A$ via $f$ è l’insieme
\begin{equation*}
    f^{-1}(A) = \set{x \mid f(x) \in A}
\end{equation*}
\begin{enumerate}[label=(\alph*) ]
    \item dare un esempio di una funzione parziale calcolabile $f$ e di un insieme ricorsivo $A$
    tale che $f^{-1}(A)$ è r.e.  ma non ricorsivo
    \item dimostrare che per ogni funzione parziale calcolabile $f$, se $A$ è r.e. allora anche
    $f^{-1}(A)$ è r.e
\end{enumerate}

Soluzione:
\begin{enumerate}[label=(\alph*) ]
    \item Basta prendere un semidecisore per un insieme r.e. non ricorsivo e $\Nat$. 
    \item Possiamo definire $s_{f^{-1}(A)}$ a partire da $s_{A}$:
    \begin{equation*}
        s_{f^{-1}(A)}(x) = s_{A}(f(x))
    \end{equation*}
\end{enumerate}

Approfondimento:

Cosa posso dire della controimmagine di un insieme $A$ attraverso una funzione $f$? 

\begin{table}[h]
    \centering
    \begin{tabular}{|c|c|c|}
    \hline
    \diagbox{$f$}{$A$} & Ricorsivo & R.E.\\
    \hline
    totale & $f^{-1}(A)$ è Ricorsivo & $f^{-1}(A)$ è r.e. \\
    \hline
    parziale & $f^{-1}(A)$ è r.e. & $f^{-1}(A)$ è r.e. \\
    \hline
    \end{tabular}
\end{table}

Questo vale in generale. Per alcuni casi particolari potrei avere situazioni più particolari. Ad
esempio funzione parziale e $A$ ricorsivo e controimmagine ricorsiva.

E dell'immagine cosa possiamo dire? L'immagine di un insieme r.e. attraverso $f$ rimane r.e.
(un'ovvietà). E se $f$ è totale e $A$ è ricorsivo? Non posso dire con certezza che l'immagine
sarà ricorsiva. Se ad esempio prendo la funzione di enumerazione di un insieme r.e. ottengo
un'immagine r.e. a partire da un insieme ricorsivo ($\Nat$).

\subsection{Esercizio 2}

È possibile enumerare ogni insieme r.e. infinito mediante una funzione di enumerazione crescente?
Motivare adeguatamente la risposta.

Soluzione:

La funzione di enumerazione di un insieme r.e. può sempre essere iniettiva (\ie posso enumerare
senza ripetizioni), ma non può essere mai crescente. Questo perchè se posso enumerare in maniera
crescente un insieme $A$ ho che $A$ è ricorsivo. Dato che esistono insiemi r.e. non ricorsivi ho che
non posso enumerare tutti gli insiemi r.e. mediante una funzione crescente.

\subsection{Esercizio 3}

Data una funzione di enumerazione $f$ totale e calcolabile, la funzione $\textit{count}_{f}(n)$ conta quante
volte $n$ compare nella enumerazione, fino ad un massimo di $100$ (se $n$ compare più di $100$ volte, l’output
è $100$):
\begin{equation*}
    \textit{count}_{f}(n) = \min(100,|\set{x \mid f(x) = n}|)
\end{equation*}
È possibile calcolare $\textit{count}_{f}$?

Soluzione:

La specifica di $\textit{count}_{f}$ è totale, vediamo se esiste un programma che la rispetti.
Intuitivamente sembrerebbe di no.

Posso scegliere la mia $f$ come voglio per mostrare un controesempio nella mia dimostrazione che
$\textit{count}_{f}$ non è calcolabile.

Scegliamo per $f$ la funzione di enumerazione di $K$. Abbiamo che
\begin{equation*}
    count_{e_{K}}(n) = 0 \iff n \notin K.
\end{equation*}

Quindi se $\textit{count}_{f}$ potrei risolvere la terminazione diagonale. Ma questo è assurdo, da
cui la non calcolabilità di $\textit{count}_{f}$.

\section{Esercizio 4}

È possibile  calcolare  il  più  piccolo  input $x$ su cui un programma da un output maggiore o
uguale di $x$?

Soluzione:

Se $\phii$ fosse totale non avrei problemi, visiterei progressivamente i miei input finché non trovo
il più piccolo per cui vale quella proprietà.

La congettura è che $g$ non sia calcolabile. Costruiamo una funzione $\phi_{h(i)}$ tale che $g$ non sia
calcolabile per $\phi_{h(i)}$.

Sia $f(i,x)$ tale che:
\begin{equation*}
    f(i,x) =
    \begin{cases}
        \case{0}{se $x = 0$ e $\phii(i) \converges$}\\
        \case{1}{se $x \geq 1$}
    \end{cases}
\end{equation*}

Per s-m-n ho $h$ che mi curryfica la mia funzione. Ora, cosa posso dire per $g(h(i))$?

\begin{equation*}
    g(h(i)) =
    \begin{cases}
        \case{0}{se $\phii(i) \converges$}\\
        \case{1}{se $\phii(i) \diverges$}\\
    \end{cases}
\end{equation*}

Da cui la non calcolabilità di $g$.

\subsection{Esercizio 5}

Classificare il seguente insieme:
\begin{equation*}
    A = \set{i \mid \text{$\phii$ è una funzione (parziale) periodica}}
\end{equation*}

Soluzione:

In questo tipo di esercizi posso usare i meta-teoremi di Rice e Rice-Shapiro.

Abbiamo che $A$ è compatto ma non monotono. Infatti se una funzione $phi_{i}$ ha un periodo $T$ per
un certo input $k$ su cui diverge possiamo restingere la funzione alla funzione sempre divergente
che ha un periodo $T$ su qualsiasi input. Per quanto riguarda la monotonia abbiamo che la funzione
sempre divegente, che è periodica, può essere estesa ad una funzione non periodica.

Approfondimento:

Il seguente insieme è compatto, monotono non r.e.:

\begin{equation*}
    A = \set{ i \mid W_{i} \cap \comp{K} \not= \emptyset} = \set{ i \mid dom(\phii) \cap \comp{K}
    \not= \emptyset } = \set{ i \mid \exists x, \phii(x) \converges \land \phi_{x}(x) \diverges}
\end{equation*}

Se estendo una funzione $\phii$, con $i \in A$, avrò sempre intersezione non vuota, quindi vale la
monotonia. Se l'intersezione è diversa dal vuoto ho che, per ogni funzione $\phii$ con $i \in A$,
esiste un $x$ che appartiene all'intersezione. Posso restringermi a quell'$x$ e ottengo una funzione
il cui indice sta ancora in $A$, e quindi vale la compattezza.

$A$ non è r.e. Dato un numero $x$ esiste sempre una funzione totale calcolabile tale che $W_{h(x)}
= \set{x}$ (si costruisce con s-m-n). Ora, $h(x) \in A \iff W_{h(x)} = \set{x} \cap \comp{K} \not=
\emptyset \iff x \in \comp{K}$.

In generale per dimostrae che un insieme $A$ non è r.e. posso provare a dimostrare che $\comp{K}
\leq A$. Per mostrare che $A$ non è ricorsivo cerco in genere una riduzione da $K$ ad $A$ ($K \leq
A$). Va tuttavia ricordato che non tutti gli insieri r.e. sono completi, quindi questa seconda
procedura non vale per tutti gli insiemi r.e. non ricorsivi.

\section{Altri esercizi}

\subsection{Esecuzione parallela}

Vediamo una definizione del parallelo, ovvero a dare una definizione più precisa di cosa significa
lanciare in parallelo due programmi.

Definiamo:
\begin{equation*}
    (f || g)(x) =
    \begin{cases}
        \case{\diverges}{se $f(x) \diverges$ e $g(x) \diverges$} \\
        \case{f(x)}{se $f(x) \converges$ e $g(x) \diverges$} \\
        \case{g(x)}{se $f(x) \diverges$ e $g(x) \converges$} \\
        \case{\max\set{f(x),g(x)}}{se $f(x) \converges$ e $g(x) \converges$}
    \end{cases}
\end{equation*}

Questa funzione è calcolabile? Intuitivamente no: supponiamo che lanciando il nostro programma $f$ 
termini. Non possiamo determinare se $g$ divergerà, quindi non saremo mai certi di cosa restituire.

Più precisamente, scegliamo per $f$ il semidecisore di $K$ e per $g$ la funzione costante $\bm{0}$.
Abbiamo che
\begin{equation*}
    (f || g)(x) =
    \begin{cases}
        \case{1}{se $\phix(x)\converges$}\\
        \case{0}{se $\phix(x)\diverges$}\\
    \end{cases}
\end{equation*}

Quindi la funzione non è calcolabile.

\subsection{Estensione totale di funzioni parziali}

Sarebbe bello se, dato un programma qualsiasi, ne esistesse una estensione totale. In questo modo
potrei lavorare sempre con l'estensione e avrei una funzione totale.

La domanda è, può una data funzione $\phii$ essere estesa a una funzione totale calcolabile? La
risposta è in generale no. Esistono funzioni parziali per cui non può esistere una estensione
totale.

La funzione che consideriamo, tra le tante, è $f(x) = \phix(x) + 1$. Sia $\hat{f}$ una estensione
totale di $f$ con indice $m$ ($\phim$). Abbiamo che $\phim(m) \converges$, essendo $\phim$ totale,
ed $f(m)$ deve di conseguenza convergere, per sua definizione. Ora, $\hat{f}(m)$ dovrebbe evere lo
stesso valore di $f(m)$, essendo una sua estensione. Ma allora avremmo $\phim(m) = f(m) = \phim(m) +
1$, il che è assurdo.

Consideriamo l'insieme $A = \set{i \mid \phii \text{ è estendibile}}$. Come lo classifichiamo? Essendo una
proprietà estensionale possiamo applicare i nostri meta-teoremi. Abbiamo inoltre, grazie al
risultato precedente, che esistono funzioni non estendibili (il che non è banale). Ora
immediatamente per Rice so che $A$ non è ricorsivo. È sicuramente compatto, dato
che $f_{\emptyset}$ è estendibile banalmente. Non è invece monotono, dato che se estendo la funzione
$f_{\emptyset}$ con la funzione $f(x) = \phi_{x}(x) + 1$  ho che $f$ non è più estendibile per il
risultato precedente.

\subsection{Classificazione di un insieme non estensionale}

Supponiamo di voler classificare l'insieme $A = \set{i \mid \phii(i) = i}$. Non essendo la proprietà
caratterizzante estensionale non posso applicare Rice o Rice-Shapiro. Ho che la proprietà è
sicuremante semidecidibile, dato che il semidecisore è il programma che, lanciando $\phii$ su input
$i$, mi dica 1 se ho output $i$, 0 altrimenti e che diverge se $\phii(i)$ diverge.

È ricorsivo? Se abbiamo il sospetto che non lo sia possiamo provare a fare una riduzione da $K$ ad
$A$.

Consideriamo la funzione binaria $g(i,x)$:
\begin{equation*}
    g(i,x) =
    \begin{cases}
        \case{x}{se $\phii(i) \converges$} \\
        \case{\diverges}{altrimenti}
    \end{cases}
\end{equation*}

Per s-m-n esiste $h$ totale calcolabile tale che $\phi_{h(i)}(x) = g(i,x)$. Ora, $h(i)$ è una buona funzione di
riduzione?

Abbiamo che:
\begin{itemize}
    \item $i \in K \implies \phi_{h(i)}(h(i)) = h(i) \implies h(i) \in A$
    \item $i \notin K \implies \phi_{h(i)}$ diverge sempre, in particolare $\phi_{h(i)}(h(i)) \not=
    h(i)$, essendo $h$ totale $ \implies h(i) \notin K$
\end{itemize}

Abbiamo quindi che $A$ non è ricorsivo.

\section{Differenza tra un insieme r.e. ed un insieme finito}

Sia $A$ r.e. non ricorsivo. Sia $B$ finito. Consideriamo $A \cup B$. Vogliamo dimostrare che è r.e.
ma non ricorsivo.

L'ipotesi di finitezza di $B$ è fondamentale. Se avessi avuto solo $B$ ricorsivo avrei potuto prendere
$\Nat$ e ottenere un insieme ricorsivo.

Supponiamo di avere il decisore per $A \cup B$, $c_{A \cup B}(n)$. Quanto è diversa questa funzione dal
semidecisore di $A$? È diversa per un sottoinsieme di $B$, ovvero un numero finito di punti.

Consideriamo i punti in $B \setminus A = \set{b_{1},\dotsc,b_{k}}$. Posso ora costruire un decisore
per $A$ nel seguente modo:
\begin{python}
def $c_{A}(x)$:
    if $x = b_{1}$:
        return 0
    else if $x = b_{2}$:
        return 0
    $\vdots$
    else:
        return $c_{A \cup B}(x)$
\end{python}

Ma quindi se $A \cup B$ fosse decidibile anche $A$ lo sarebbe. Ma questo è assurdo.

Non è un problema la non calcolabilità di $B \setminus A$, l'importante è l'esistenza di questi
punti.

\section{Remark finale sulla calcolabilità di certe funzioni}

Un conto è quando non esiste un programma, un conto è quando un programma esiste ma
non ho modo di calcolarlo a partire da certe informazioni. Nel primo caso sto trattando un problema
non calcolabile, come la terminazione generale, nel secondo caso il problema è calcolabile ma non è
calcolabile l'indice del programma che mi calcola il mio problema sulla base di certe informazioni.

Ad esempio, prendiamo la seguente famiglia di funzioni:
\begin{equation*}
    g(x) =
    \begin{cases}
        \case{1}{se $\phii(i) \converges$} \\
        \case{0}{se $\phii(i) \diverges$} \\
    \end{cases}
\end{equation*}

Qui $i$ è una variabile libera. Fissato un $i$ la funzione $g$ è calcolabile. Se ho che $\phii(i)
\converges$ allora $g$ corrisponde alla costante $\bm{1}$. Altrimenti corrisponde alla costante
$\bm{0}$. Queste sono due funzioni calcolabili. Quello che non posso calcolare sono gli indici $i$
per cui $g$ corrisponde a una a all'altra funzione.

In particolare non posso calcolare:
\begin{equation*}
    g(i,x) =
    \begin{cases}
        \case{1}{se $\phii(i) \converges$} \\
        \case{0}{se $\phii(i) \diverges$} \\
    \end{cases}
\end{equation*}
analogo della funzione precedente ma in cui $i$ è legata.
