%% tex/fixedpoint.tex
%% Copyright 2019 Andrea Berlingieri
%
% This work may be distributed and/or modified under the
% conditions of the LaTeX Project Public License, either version 1.3
% of this license or (at your option) any later version.
% The latest version of this license is in
%   http://www.latex-project.org/lppl.txt
% and version 1.3 or later is part of all distributions of LaTeX
% version 2005/12/01 or later.
%
% This work has the LPPL maintenance status `maintained'.
%
% The Current Maintainer of this work is Andrea Berlingieri.
%
% This work consists of all files listed in manifest.txt
\chapter{Teorema del punto fisso}

\section{Il teorema del punto fisso di Kleene}

\begin{thm}
    Per ogni funzione totale calcolabile $f$ esiste $m$ tale che 
    \begin{equation*}
        \phi_{f(m)} = \phi_{m}
    \end{equation*}
\end{thm}
\begin{proof}
    Sia $g$ così definita:
    \begin{equation*}
        g(x,y) = \phi_{f(\phix(x))}(y)
    \end{equation*}
    Per s-m-n esiste $h$ tale che $\phi_{h(x)}(y) = g(x,y)$, per $x,y$ arbitari. Sia $p$ un indice
    per $h$ e poniamo $m = \phip(p) = h(p)$. $m$ è sicuramente definito, essendo $h$ totale.
    Abbiamo ora che, per ogni $y$:
    \begin{equation*}
        \phim(y) = \phi_{\phip(p)}(y) = \phi_{h(p)}(y) = g(p,y) = \phi_{f(\phip(p))}(y) =
        \phi_{f(m)}(y)
    \end{equation*}
\end{proof}

Se prendiamo per $f$ la funzione successore ho per forza che nel mio ordinamento avrò due programmi
adiacenti che si comportano nella stessa maniera. In questo modo non è possibile creare un
ordinamento tale che questo non si verifichi. La ragione fondamentale è che non ho controllo sul
comportamento dei programmi quando vado ed enumerarli.

Punti fissi e ricorsione sono fondamentalmente la stessa cosa vista da due punti di vista diversi.

La ricorsione è legata alla possibilità che nel corpo del mio programma sia visibile la funzione
che sto definendo.

Supponiamo di voler costruire un programma che stampi se stesso, ovvero che stampi il suo indice.
Per semplicità lo vogliamo costante, ovvero tale che per ogni input stampi il suo indice: $\forall
x, \phip(x) = p$.

Prendiamo $g(i,x)$ così definita:
\begin{equation*}
    g(i,x) = i
\end{equation*}

Per s-m-n otteniamo una classe di programmi costanti $\phi_{h(i)}(x)$. Questi però stampano sempre
$i$, anche se $h(i)$ è diverso da $i$, quindi non vanno bene. Sfruttando il teorema del punto fisso
abbiamo però che esiste $\phip(x) = \phi_{h(p)}(x) = p$. Di conseguenza è sufficiente prendere il
punto fisso di $h$, $p$.

Se io voglio realizzare un comportamento ricorsivo posso farlo con il teorema del punto fisso di
Kleene. Supponiamo di voler una funzione calcolabile $\phip$ tale che $\phip(x) = f(\phip(x))$, per
qualche $f$. Si può? Sì. Abbiamo che $g(i,x) = f(\phii(x))$. Ma ora per s-m-n abbiamo che questo
corrisponde a $\phi_{h(i)}(x)$. Ma ora abbiamo che $\phip(x) = \phi_{h(p)}(x) = f(\phip(x))$, se
prendiamo come $p$ il punto fisso di $h$.

In generale per punto fisso di una funzione intendiamo un input $x$ tale che $f(x) = x$. Il nostro
senso è un pò particolare perché è un punto fisso estensionale.

Sfruttando questo risultato vogliamo costruire un programma che dato $i$ mi restituisca qualcosa
diverso dalla funzione $\phii$ per almeno un input, ovvero vogliamo $i$ tale che:
\begin{equation*}
    \exists x, g(i,x) \not= \phii(x).
\end{equation*}

Proviamo con $g(i,x) = \phii(x) + 1$. Per s-m-n esiste $h$ tale che $\phi_{h(i)}(x)$ identico a $g$. 
Funziona? No, se $i$ è un indice per la funzione sempre divergente. Si può fare di meglio? No, per
il teorema del punto fisso. Per qualunque trasformazione di programmi ho un punto fisso tale che
$\phi_{h(m)} = \phim$. Non ho speranza di fare una modifica effettiva ed uniforme tale che il mio
programma sia diverso da quello di partenza, e questo vale per ogni programma.

Vediamo una nuova dimostrazione del teorema di Rice basata sul teorema del punto fisso di Kleene. È
concettualmente diversa da quella vista in precedenza.

Sia $A$ un insieme estensionale. $A$ è ricorsivo sse è banale. Supponiamo per assurdo che $A$ sia
ricorsivo ma non banale, ovvero esista $i \in A$ e $j \in \comp{A}$. Posso definire $h$ tale che:
\begin{equation*}
    h(x) = 
    \begin{cases}
        \case{j}{se $x \in A$}\\
        \case{i}{se $x \notin A$}
    \end{cases}
\end{equation*}

Questa funzione sarebbe calcolabile per la ricorsività di $A$. Per il teorema del punto fisso,
essendo $h$ totale e calcolabile, deve esistere $m$ tale che $\phim = \phi_{h(m)}$. Ci chiediamo ora,
dove sta $m$? Se $m \in A$ allora stanno in $A$ anche tutti i programmi che si comportano come $m$,
tra cui anche $h(m)$. Ma se $m \in A$ allora $h(m) = j$, quindi $h(m) \notin A$. Se invece $m \notin
A$ allora $h(m) \notin A$. Ma per definizione di $h$ si ha che $h(m) = i \in A$. Ecco la mia
contraddizione.

Tutte le volte che cerchiamo programmi che dipendono, intensionalmente o meno, dall'indice del
programma che sto cercando devo fare riferimento al teorema del punto fisso.

\section{Il secondo teorema del punto fisso}

Pensiamo ora ad una funzione binaria $f(x,y)$. Possiamo generalizzare il teorema del punto fisso:
\begin{thm}
    $\forall f^{2}$ totale calcolabile $\exists s$ totale calcolabile tale che $\forall y, \phi_{f(s(y),y)} = \phi_{s(y)}$
\end{thm}
\begin{proof}
    Definiamo la funzione $g$ come
    \begin{equation*}
         g(x,y,z) = \phi_{f(\phix(x),y)}(z)
    \end{equation*}
    Per s-m-n esiste $h$ totale calcolabile tale che:
    \begin{equation*}
        \phi_{h(x,y)}(z) = g(x,y,z)
    \end{equation*}
    Definiamo ora $g'(y,x) = h(x,y)$. Per s-m-n esiste $r$ totale calcolabile tale che
    \begin{equation*}
        \phi_{r(y)}(x) = g'(y,x) = h(x,y)
    \end{equation*}
    Definiamo ora la funzione $s(y) = \phi_{r(y)}(r(y))$. Abbiamo che, per ogni $z$:
    \begin{equation*}
        \phi_{s(y)}(z) = \phi_{\phi_{r(y)}(r(y))}(z) = \phi_{h(r(y),y)} =
        \phi_{f(\phi_{r(y)}(r(y)),y)}(z) = \phi_{f(s(y),y)}(z)
    \end{equation*}
\end{proof}
