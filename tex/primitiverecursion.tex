%% tex/primitiverecursion.tex
%% Copyright 2019 Andrea Berlingieri
%
% This work may be distributed and/or modified under the
% conditions of the LaTeX Project Public License, either version 1.3
% of this license or (at your option) any later version.
% The latest version of this license is in
%   http://www.latex-project.org/lppl.txt
% and version 1.3 or later is part of all distributions of LaTeX
% version 2005/12/01 or later.
%
% This work has the LPPL maintenance status `maintained'.
%
% The Current Maintainer of this work is Andrea Berlingieri.
%
% This work consists of all files listed in manifest.txt
\chapter{Il formalismo primitivo ricorsivo}

\section{Ricorsione primitiva}

Nei linguaggi di programmazione siamo abituati all'autoreferenzialità, o ricorsione. In generale
bisogna fare attenzione.

Esistono forme esplicite ed implicite di autoreferenziazione. Per le seconde ho un oggetto che ha
una sezione universale nel suo scope e tale che l'oggetto stesso ci rientra. È una sorta di ordine
superiore.

Ad esempio:
\begin{quote}
    ``Tutte le sentenze universali sono false''
\end{quote}

Questa frase è implicitamente autoreferenziale. Nel suo raggio di applicabilità include se stessa.

Per le formule aperte non ha senso parlare di verità. Si può parlare solo di verità in caso di
formule chiuse, ovvero sentenze. I teoremi sono tutti chiusi. La sentenza sopra è dimostrabilmente falsa.

Ci chiediamo se la definibilità di una funzione implichi la sua calcolabilità e anche se vale il 
viceversa. Sappiamo già che la nozione di definibilità è legata al linguaggio che utilizzo.

Possiamo dare una definizione precisa di calcolabilità? È una problematica delicata, legata al
sistema di calcolo che utilizzo e alla definizione formale di algoritmo, che non è banale.

%// TODO DEVE STARE PROPRIO QUA QUESTA AFFERMAZIONE?
%È facile dimostrare che il while è più espressivo del for dato che è possibile implementare il
%for con un while.

Ci chiediamo, vogliamo davvero avere un loop infinito? In certi casi sì, ad esempio per stampare
una lista di numeri primi ed interromperla quando voglio. Ma non è comune, molto spesso ci basta
una iterazione determinata.

Le funzioni dell'informatica sono di una categoria particolare: sono funzioni parziali. Le funzioni
totali calcolabili sono un sottoinsieme delle funzioni parziali.

Ci poniamo il problema di giustificare l'inclusione, nei nostri linguaggi di programmazione, di
costrutti che possono causare divergenza. Hanno un'utilità che giustifica il rischio di non
terminazione dei programmi.

Noi siamo abituati a scrivere funzioni ricorsive. Ad esempio, $\textit{Fact}(n+1) =
(n+1)*\textit{Fact}(n)$, assieme al caso base. Dal punto di vista matematico questo oggetto è
strano. Non si può definire una cosa in funzione di se stessa. C'è modo di definire in una maniera
più sensata a livello matematico una funzione ricorsiva, che sottintende una procedura? È una
problematica interessante.

La ricorsione primitiva è un tipo di ricorsione simile a quella finora vista ma con dei limiti
semantici. In particolare possiamo dare il vincolo: nel corpo della definizione di $f(n+1)$ ci si
può richiamare ricorsivamente solo su $n$. È equivalente dal punto di vista dell'espressività un
indebolimento di questo vincolo, in cui si limitano invece le chiamate ricorsive a oggetti più
piccoli di $n+1$.

La ricorsione primitiva, per vincoli strutturali, garantisce la terminazione. Siamo interessati a
questa classe di funzioni perché c'è un teorema che dice che le funzioni definibili in questo modo
sono esattamente quelle definibili con un for.

Prendiamo come primitive alcune funzioni nel nostro linguaggio primitivo ricorsivo $\Lang$:
\begin{enumerate}
    \item le funzioni costanti:
        \begin{equation*}
            c^{k}_{m}(x_{1},x_{2},\dotsc,x_{k}) = m \quad \text{$m$ in $\Nat$}
        \end{equation*}
    \item le proiezioni:
        \begin{equation*}
            \pi_{i}^{k}(x_{1},x_{2},\dotsc,x_{k}) = x_{i} \quad \text{per qualche $1 \leq i \leq k$}
        \end{equation*}
    \item la funzione successore:
        \begin{equation*}
            s(x) = x + 1
        \end{equation*}
\end{enumerate}

Ammettiamo inoltre due schemi composizionali per definire nuove funzioni a partire da quelle già
esistenti:
\begin{enumerate}
    \item Composizione: se $h : \Nat^{n} \to \Nat$ e $g_{1},g_{2},\dotsc,g_{n}:\Nat^{k} \to \Nat$
    appartengono a $\Lang$, anche la funzione $f: \Nat^{k} \to \Nat$ così definita:
    \begin{equation*}
        f(\vec{x}) = h(g_{1}(\vec{x}),g_{2}(\vec{x}),\dotsc,g_{n}(\vec{x}))
    \end{equation*}
    appartiene a $\Lang$, con $\vec{x} = (x_{1},x_{2},\dotsc,x_{k})$.
    \item Ricorsione primitiva: se $g: \Nat^{k} \to \Nat$,$h: \Nat^{k+2}\to \Nat$ appartengono a
    $\Lang$, allora che $f: \Nat^{k+1} \to \Nat$, così definita:
    \begin{equation*}
        f(n,\vec{x}) =
        \begin{cases}
            \case{f(0,\vec{x}) = g(\vec{x})}{}\\
            \case{f(y+1,\vec{x}) = h(y,f(y,\vec{x}),\vec{x})}{}\\
        \end{cases}
    \end{equation*}
    appartiene a $\Lang$, con $\vec{x} = (x_{1},x_{2},\dotsc,x_{k})$.
\end{enumerate}

Quando ho più strade di espansione nei sistemi di riscrittura il problema di capire se la strada
che scelgo è influente è il problema della confluenza, legato alla terminazione dell'espansione.

Nell'esecuzione ci sono dei problemi che non abbiamo indicato esplicitamente, come la valutazione
per valore e per nome, l'ordine di valutazione delle espressioni, i side effect, ecc.

%Non è chiaro cosa calcoli la funzione. È possibile semplificare il codice mediante inlining.

L'idea delle funzioni ricorsive primitive è che abbiamo un argomento su cui facciamo la ricorsione
ed una serie di parametri aggiuntivi.

Lavorare su una ``sottostruttura'' dell'input in ricorsione garantisce la terminazione della chiamata.
La ricorsione primitiva è un sottocaso della ricorsione strutturale. Con la seconda ci si richiama
solo su sottostrutture strette.

\subsection{Esempi di funzioni primitive ricorsive}

Vediamo ora alcuni esempi di definizioni di funzioni comuni nel formalismo primitivo ricorsivo.

\begin{equation*}
    \textit{add}(x,y) = 
    \begin{cases}
        \textit{add}(0,y) = y \\
        \textit{add}(x+1,y) = \textit{succ}(\textit{add}(x,y))
    \end{cases}
\end{equation*}

\begin{equation*}
    \textit{mult}(x,y) = 
    \begin{cases}
        \textit{mult}(0,y) = 0 \\
        \textit{mult}(x+1,y) = \textit{add}(\textit{mult}(x,y),y)
    \end{cases}
\end{equation*}

Il meccanismo della definizione di funzioni per ricorsione primitiva è naturale. Molte funzioni
sono strutturate in maniera ricorsiva.

\begin{equation*}
    \textit{fact}(x) =
    \begin{cases}
        \textit{fact}(0) = 1 \\
        \textit{fact}(x+1) = (x+1)*\textit{fact}(x)
    \end{cases}
\end{equation*}

%La calcolabilità la definiamo sui numeri naturali, interi positivi.
%Per la differenza fai un preambolo del tipo "Vediamo due definizioni della funzione sub, una
%sbagliata ..."

La seguente funzione mi restituisce 1 se l'input è 0 e 0 altrimenti.
\begin{equation*}
    \textit{test}(x) = 
    \begin{cases}
        \textit{test}(0) = 1 \\
        \textit{test}(x+1) = 0
    \end{cases}
\end{equation*}

Proviamo a definire la funzione differenza. Ricordiamo che abbiamo definito la calcolabilità sui
numeri naturali, interi positivi. Di conseguenza non possiamo calcolare la differenza, ad esempio,
tra 3 e 7. Noi vogliamo una funzione che calcoli $a - b$ se $a > b$. In tutti gli altri casi
calcoliamo 0.

Una prima definizione naïve di $a - b$ è la seguente:
\begin{equation*}
    \textit{sub}(a,b) =
    \begin{cases}
        \textit{sub}(0,b) = 0 \\
        \textit{sub}(a+1,b) = \textit{succ}(\textit{sub}(a,b))
    \end{cases}
\end{equation*}

Questa è sbagliata rispetto alla nostra specifica. Per $sub(1,1)$, ad esempio, restituisce 1. Una
scelta migliore sarebbe andare in ricorsione su b.

Una delle scelte da fare è su cosa andare in ricorsione. Può essere uno dei parametri oppure un
nuovo valore costruito a partire dai parametri.

Proviamo con la seguente definizione:
\begin{equation*}
    \textit{sub}(a,b) =
    \begin{cases}
        \textit{sub}(a,0) = a \\
        \textit{sub}(a,b+1) = \textit{pred}(\textit{sub}(a,b))
    \end{cases}
\end{equation*}
dove \textit{pred} restituisce 0 per 0 e $x-1$ in ogni altro caso.

Questa definizione è un pò borderline, perché andrebbe dimostrato che cambiando il parametro su
cui vado in ricorsione ottengo un formalismo equivalente a quello introdotto. Tuttavia ciò è
possibile perciò la definizione rientra nel formalismo primitivo ricorsivo.

Vogliamo ora una funzione che restituisca 1 se i parametri sono uguali, 0 altrimenti.

\begin{equation*}
    \textit{comp}(n,m) = test(\textit{add}(\textit{sub}(n,m),\textit{sub}(m,n)))
\end{equation*}

Benchè sembri limitante è veramente potente questo tipo di ricorsione.

Quando parliamo di predicati intendiamo funzioni che restituiscano un booleano.

Supponiamo di saper calcolare un certo predicato $P$. Possiamo calcolare anche la sua negazione.

Data la funzione caratteristica di $P$, $c_{P}$, possiamo calcolare la funzione caratteristica di
$\comp{P}$: $c_{\comp{P}}(x) = 1 - c_{P}(x)$.

Date $c_{P}$ e $c_{Q}$ abbiamo che $c_{P \land Q}(x) = c_{P}(x) *
c_{Q}(x)$.

È possibile definire $c_{P \lor Q}$ con le leggi di de Morgan: $A \lor B = \comp{\comp{A \lor B}} =
\comp{\comp{A} \land \comp{B}}$. Un altro modo è usano la funzione normalizzazione, che normalizza i
numeri a 0 e 1:
\begin{equation*}
    c_{P \lor Q}(x) = \textit{norm}(c_{P}(x) + c_{Q}(x))
\end{equation*}

Dati dei predicati possiamo quindi calcolare i connettivi logici tra di loro nel nostro formalismo.
Passiamo ora al discorso dei quantificatori.

Per calcolare il quantificatore esistenziale dovrei avere una procedura del genere:

% python listing?
\begin{python}
x = 0
while $\lnot P(x)$:
    x = x + 1
\end{python}

Questo potrebbe ciclare all'inifinito se l'esiste non vale. Abbiamo un problema duale con il
quantificatore universale.

È evidente che c'è un problema ma non è detto che questo non sia insormontabile.

Quello che sappiamo senza dubbio calcolare è la quantificazione limitata, o bound. L'idea è che
ho un upper bound finito alla mia quantificazione. Calcoleremmo quindi $g(n) = \exists x \leq n,
P(x)$.

Ovviamente possiamo calcolare la quantificazione limitata con la ricorsione primitiva. Nel caso del
quantificatore universale:

\begin{align*}
    f(0) &= P(0) \\
    f(n+1) &= P(n+1)*f(n)
\end{align*}

Tanti problemi aperti dell'aritmetica sono legati alla quantificazione: sapere se esiste un numero
che ripetti tale proprietà o se una tale proprietà vale per tutti i numeri.

È anche una maniera per approcciarsi ai problemi. Domandarsi se c'è un bound permette di rendere
l'algoritmo più efficiente e certamente terminante.

\section{Test di primalità}

Vediamo ora un altro predicato interessante, il test di primalità.

Perché escludiamo 1 dai numeri primi? Perché uno dei risultati più importanti dell'aritmetica è
la fattorizzazione unica, che vale per tutti i numeri escludendo 1. Per mantenere quel teorema 1
viene escluso.

Definiamo la proprietà $P(x)$:

\begin{equation*}
    x > 1 \land \forall z (z | x \implies (z = 1 \lor z = x))
\end{equation*}

$z | x$ è notazione per $z$ divide $x$ (più precisamente, $z$ è un fattore di $x$).

\begin{equation*}
    z | x := \exists a, z * a = x := \exists a \leq x, z * a = x
\end{equation*}

Possiamo porre un bound anche al test di primalità: possiamo fermarci ad $x$. Siamo quindi in grado di
definire (e calcolare) il test di primalità.

\section{Minimizzazione}

Vediamo ora un'altra operazione che opera su domini limitati che useremo molto spesso: la
minimiazzazione.

\begin{equation*}
    \mu x, P(x)
\end{equation*}

Possiamo vederla come uno snippet del genere:

\begin{python}
x = 0
while $\lnot P(x)$:
    x = x + 1
return x
\end{python}

Fissiamo un ordinamento e cerchiamo il primo $x$ che soddisfa $P(x)$. Quello che ci interessa alla fine
del ciclo è il valore di $x$. Questo è, sostanzialmente, cosa l'operazione di minimizzazione fa.

While è un costrutto imperativo. Noi preferiamo, per la nostra teoria della calcolabilità, un
costrutto funzionale, da cui l'introduzione del $\mu$. Il risultato di questo operatore è $x$.

Al solito col while abbiamo il problema che il while potrebbe non fermarsi mai. Quello che possiamo
certamente sperare di scrivere, e quindi calcolare, nel nostro formalismo è una forma limitata di
$\mu$. Cerchiamo il più piccolo $x$ minore di un certo $y$ su cui vale un predicato $P(x,\vec{z})$.

\begin{equation*}
    f(y,\vec{z}) = \mu x \leq y, P(x,\vec{z})
\end{equation*}

I parametri di $\vec{z}$ rappresentano parametri ulteriori che possono essere utili.

Cosa vogliamo restituire se non troviamo $x$ nell'intervallo fissato? Dobbiamo restituire un valore
e ne possiamo restituire uno qualunque. La cosa più naturale è restituire $y + 1$, se $y$ è il
mio upper bound.

Vogliamo trovare un modo per scrivere questa operazione, non necessariamente per calcolarla
efficientemente.

Definiamo prima il predicato $R$:
\begin{equation*}
    R(y,\vec{z}) = \forall x \leq y, \lnot P(x,\vec{z})
\end{equation*}

Questo mi dice ``fino a $y$ non ho trovato il minimo'', con $y$ compreso.

$R$, come funzione, è una costante di valore 1 fino all'$y_{0}$ minimo per cui vale
$P(y_{0},\vec{z})$. Da lì in poi il suo valore diventa costantemente 0.

Possiamo ora scrivere $\mu x$:

\begin{equation*}
    \mu x \leq y, P(x,\vec{z}) = \sum_{w \leq y}\lnot R(w,\vec{z})
\end{equation*}

Possiamo definirlo in un altro modo. Prendiamo in considerazione il predicato $M$:

\begin{equation*}
    M(x,\vec{z}) = P(x,\vec{z}) \land \forall y < x, \lnot P(y,\vec{z})
\end{equation*}

Questo predicato mi dice ``$x$ è il più piccolo valore per cui vale $P$''. Come faccio però a trovare
$x$? Si può moltiplicare $x$ per $M(x,\vec{z})$. Dato che dobbiamo testare tutti gli $x \leq y$, abbiamo
che $\mu$ può essere espresso come:
\begin{equation*}
    \mu x \leq y, P(x,\vec{z}) = \sum_{x \leq y} x \cdot M(x,\vec{z})
\end{equation*}

\section{Generazione numeri primi}

Come possiamo trovare il più piccolo numero primo successivo ad un numero $i$? Con la seguente
funzione, ad esempio:
\begin{equation*}
    \Pi(i) = 
    \begin{cases}
        \Pi(0) = 2 \\
        \Pi(x + 1) = \mu p. \textit{prime}(p) \land p > \Pi(x)
    \end{cases}
\end{equation*}

Cosa manca? Bisogna dare un bound alla minimizzazione. C'è un teorema che dice che è sempre
possibile trovare un numero primo tra $n$ e $2n$. Il bound che possiamo dare è $2\Pi(x)$.

L'importante è dare un bound. C'è un altro bound, molto più grande, che andrebbe bene comunque:
$\Pi(x)!$.

Come si dimostra l'infinità dei numeri primi? Supponiamo che siano finiti e siano $p_{1}, p_{2},
\dotsc, p_{n}$. Prendiamo il numero $p_{1}\cdot p_{2}\cdots p_{n} + 1$. Questo numero non è
divisibile per nessun $p_{i}$ della mia enumerazione. Quindi o è un numero primo oppure i suoi
fattori non fanno parte di quella lista. In ogni caso ho un assurdo.

È una dimostrazione bella. La tecnica è analoga alla diagonalizzazione: costruisco un nuovo
elemento da quelli di una lista che prova il mio assurdo.

\section{Ricorsione multipla}

Consideriamo una possibile codifica del piano e consideriamo la coppia $<n,m>$. Non siamo troppo
interessati alla codifica della coppia in sè, ma alle funzioni che mi restituiscono le componenti
della coppia.

Abbiamo che, in generale, le componenti sono $\leq$ della (codifica della) coppia: $n \leq <n,m>$ e
$m \leq <n,m>$.

Supponiamo di voler calcolare $\pi_{1}(x)$, ovvero la prima proiezione della coppia $x$. Partiamo
dalla seguente definizione:
\begin{equation*}
    \pi_{1}(x) = \mu n, \exists m, <n,m> = x
\end{equation*}

Manca un bound. Quale prendiamo? $x$:

\begin{equation*}
    \pi_{1}(x) = \mu n \leq x, \exists m \leq x, <n,m> = x
\end{equation*}

Talvolta la ricorsione può essere necessaria su più di un valore. Nel formalismo che abbiamo visto
finora non abbiamo questa possibilità.

Consideriamo la sequenza di Fibonacci:
\begin{align*}
    \textit{fib}(0) &= 1 \\
    \textit{fib}(1) &= 1 \\
    \textit{fib}(x+2) &= \textit{fib}(x+1) + \textit{fib}(x)
\end{align*}

La funzione di Fibonacci è intrinsicamente esponenziale, ma questo è il peggior metodo per
calcolarlo. Siamo esponenziali nel numero di chiamate oltre ad esserlo nell'input. Inoltre così
com'e scritta ora non rispetta i vincoli del formalismo primitivo ricorsivo.

Qual è l'idea? Bisogna portarsi dietro un accumulatore.

Vogliamo definire la seguente funzione:
\begin{equation*}
    \textit{fibo'}(x) = <\textit{fib}(x),\textit{fib}(x+1)>
\end{equation*}

Possiamo definirla nel formalismo primitivo ricosivo nel seguente modo:
\begin{align*}
    \textit{fibo'}(0) &= <1,1> \\
    \textit{fibo'}(x+1) &= <\textit{fib}(x+1),\textit{fib}(x+2)> \\
    &= <\pi_{2}(\textit{fibo'}(x)),\pi_{1}(\textit{fibo'}(x)) + \pi_{2}(\textit{fibo'}(x))>
\end{align*}

L'unica chiamata ricorsiva che faccio è $\textit{fibo'}(x)$.

Questa funzione corrisponde all'incirca al seguente snippet:
 
\begin{python}
$acc_{0}$ = 1
$acc_{1}$ = 1 
for $i$ in range(x+1):
    tmp = $acc_{0}$
    $acc_{0}$ = $acc_{1}$
    $acc_{1}$ = tmp + $acc_{1}$
return $acc_{1}$
\end{python}

Questa tecnica è generale è prende il nome di ricorsione di coda.

Tutte le funzioni ricorsive primitive possono essere espresse mediante un for. È possibile
dimostrare anche il viceversa. Il formalismo primitivo ricorsivo è equivalente, a potere
espressivo, ai programmi scrivibili con il for, ovvero senza while e senza ricorsione generale.

\section{Funzione di Ackermann}

Vediamo ora una funzione che non è possibile scrivere nel formalismo primitivo ricorsivo. Va
immaginata come una famiglia di funzioni dove il primo parametro mi istanza la particolare funzione.
Abbiamo $\textit{ack}_{0}$, $\textit{ack}_{1}$, etc.

\begin{align*}
    \textit{ack}(0, 0, y ) &= y\\
    \textit{ack}(0, x + 1, y ) &= \textit{ack}(0, x, y ) + 1\\
    \textit{ack}(1, 0, y ) &= 0\\
    \textit{ack}(z + 2, 0, y ) &= 1\\
    \textit{ack}(z + 1, x + 1, y ) &= \textit{ack}(z, \textit{ack}(z + 1, x, y ), y )
\end{align*}

I casi sono mutalmente disgiunti. Data una tripla qualsiasi di valori solo una riga si applica.

La funzione termina? Sì, ma non è banale. L'argomento più importante della funzione è il primo.
Se il primo decresce bene. Altrimenti vado a vedere gli altri. Questo mi dà un ordinamento delle
mie chiamate ricorsive che mi da un'idea del fatto che la funzione è terminante.

Cosa calcola? $\textit{ack}_{0}$ è abbastanza banale: è la somma. $\textit{ack}_{1}$ invece
calcola il prodotto tra $x$ e $y$. $\textit{ack}_{2}$ calcola l'elevamento a potenza ($y^x$).

$\textit{ack}_{i}$ itera $\textit{ack}_{i-1}$. Il prodotto itera la somma. L'elevamento a potenza
itera la moltiplicazione la tetrazione itera l'elevamento a potenza.

La funzione, benchè terminante, ha una complessità spaventosa.

Perché non posso scriverla nel formalismo primitivo ricorsivo? Perché questa funzione cresce
troppo velocemente. Cresce più velocemente di qualsiasi funzione esprimibile col formalismo
primitivo ricorsivo. La funzione di Ackermann mi dà un bound computazionale alle funzioni
esprimibili con il formalismo primitivo ricorsivo. Di conseguenza non può essere esprimibile nello
stesso formalismo.

Se riesco ad esprimere un bound computazionale alla complessità di un programma so che non è
possibile calcolare quel bound nello stesso formalismo in cui ho espresso il mio programma.

Tutte le singole istanze di Ackermann sono scrivibili nel formalismo primitivo. Ma non posso
scrivere un programma che le esprima tutte. Il formalismo non è abbastanza parametrico.

È sufficiente  aggiungere l'ordine superiore al formalismo primitivo per poter esprimere la
funzione di Ackermann.

La funzione di Ackermann è una funzione chiaramente calcolabile, essendo esprimibile in un qualche
linguaggio di programmazione, ma non è esprimibile nel formalismo primitivo ricorsivo.

Si può dimostrare che c'è un ordinamento ben fondato e quando facciamo le chiamate ricorsive nella
funzione di Ackermann le variabili decrescono secondo questo ordine.

Un ordinamento si dice ben fondato se non esistono catene discendenti infinite.

Dire che ho un ordinamento ben fondato non è la stessa cosa di dire che di elementi minori di un
dato elemento $m$ ce ne sono una quantità finita.

Vediamo un esempio: l'ordinamento lessicografico. Considerando $\Nat \times \Nat$. Definiamo
l'ordinamento $<_{p}$ nel seguente modo: $<n_{1},m_{1}> <_{p} <n_{2},m_{2}> \iff n_{1} < n_{2} \lor
(n_{1} = n_{2} \land m_{1} < m_{2})$. Questo ordinamento è ben fondato. Vale che $\forall m, <2,m>
<_{p} <3,7>$.

Non è possibile costruire catene discendenti infinite. Proviamo a costruirne una:
$<3,7> >_{p} <3,6> ... <3,0> >_{p} <2,10^{4}>$

Arrivati qui posso ripetere il giochetto di prima finché non arrivo a 0, dopodiché dovrò
decrementare la prima componente. Di queste sequenze ne posso fare di lunghezza arbitraria ma sempre
finita. Questo ragionamento ci dà un'idea del perché la funzione di Ackermann termina (il
principio alla base della dimostrazione è lo stesso).
