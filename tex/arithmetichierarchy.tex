\chapter{La gerarchia aritmetica}

\section{Aritmetica e incompletezza}

Quando vogliamo parlare di qualcosa abbiamo bisogno di un linguaggio. Partiamo da un insieme di
segni.

Il linguaggio dell'aritmetica è il linguaggio del primo ordine basato sulla
seguente segnatura:
\begin{equation*}
    0,S,+,\cdot,=
\end{equation*}

Indichiamo con $\bar{n}$ il termine che rappresenta il numero $n$. Occhio a non fare confusione tra
termine che rappresenta un numero e numero stesso. Ad esempio 3 (il numero) vs $S(S(S(0))))$ (il
termine per il numero 3).

\begin{defn}
    $A \subseteq \Nat^{k}$ si dice aritmetico se esiste una formula aritmetica
    $\psi(x_{1},\dotsc,x_{k})$ tale che
    \begin{equation*}
        (n_{1},\dotsc,n_{k}) \in A \iff \StdNat \models \psi[\bar{n_{1}}/x_{1},\dotsc,\bar{n_{k}}/x_{k}]
    \end{equation*}
    ovvero $\psi(x_{1},\dotsc,x_{k})$ è vera sui numeri naturali. Diremo in questo caso che $\psi$ è una
    descrizione aritmetica di $A$.
\end{defn}

Ad esempio, l'insieme dei numeri primi è aritmetico, in quanto può essere
descritto dalla seguente formula aritmetica:
\begin{equation*}
    \psi(n) = n \geq 2 \land \forall y, 1 < y \land y < n \implies \forall z, y \cdot z \not= n
\end{equation*}

Non tutti gli insiemi sono aritmetici. I possibili sottoinsiemi di $\Nat$ sono più che numerabili,
non posso aspettarmi di descrivere tutti questi nell'aritmetica.

Gli insiemi aritmetici sono chiusi rispetto alle operazioni di unione, intersezione
e complementazione.

\begin{defn}
    Una funzione $f$ si dice aritmetica se il suo grafo è un insieme aritmetico.
\end{defn}

Le seguenti funzioni sono aritmetiche:
\begin{itemize}
    \item la costante 0, descritta dalla formula $\psi(y) := y = 0$;
    \item la funzione unaria successore, descritta dalla formula $\psi(x,y) := y = S(x)$;
    \item la funzione binaria somma, descritta dalla formula $\psi(x_{1},x_{2},y) := y = x_{1} + x_{2}$;
    \item la funzione binaria prodotto, descritta dalla formula $\psi(x_{1},x_{2},y) := y = x_{1} \cdot x_{2}$;
    \item la proiezione $k$-aria $\pi_{i}^{k}$, descritta dalla formula $\psi(x_{1},x_{2},\dotsc,x_{k},y) :=
    y = x_{k}$;
\end{itemize}

È ragionevole chiedersi se anche le funzioni aritmetiche sono chiuse rispetto ai metodi tipici per
comporre le funzioni: definire nuove funzioni a partire da quelle esistenti.

Il primo modo che ci può venire in mente è la composizione di funzioni.

\begin{lem}
    La composizione di funzioni aritmetiche è aritmetica.
\end{lem}
\begin{proof}
    Sia $f(x) = g(h(x))$ e siano $\psi_{g}(x,y)$ e $\psi_{h}(x,y)$ le descrizioni aritmetiche di $g$
    e $h$.

    Definiamo
    \begin{equation*}
        \psi_{f}(x,z) = \exists y, \psi_{h}(x,y) \land \psi_{g}(y,z)
    \end{equation*}

    Si può dimostrare che $\psi_{f}$ è una descrizione aritmetica di $f$.
\end{proof}

Altre operazioni di definizione di nuove funzioni tipiche sono la ricorsione primitiva e la minimizzazione.

Se abbiamo la minimizzazione (che corrisponde circa ad un while) possiamo avere subito la ricorsione
primitiva (che corrisponde ad un for).

\begin{lem}
    Una funzione definita per minimizzazione di una funzione aritmetica è ancora aritmetica.
\end{lem}
\begin{proof}
    Sia
    \begin{equation*}
        f(x) = \mu y, (g(x,y)=0)
    \end{equation*}
    e supponiamo che $\psi_{g}$ sia una descrizione aritmetica di $g$. $f$ è descritta dalla
    seguente formula:
    \begin{equation*}
        \psi_{f}(x,y) = \psi_{g}(x,y,0) \land \forall z, z < y \implies \exists m, (\psi_{g}(x,z,m)
        \land m \not= 0)
    \end{equation*}
\end{proof}

Possiamo prendere la descrizione delle funzioni come la specifica di un programma. Ci poniamo quindi
il problema di quali funzioni sono specificabili.

\begin{thm}
    Tutte le funzioni calcolabili sono aritmetiche.    
\end{thm}

Questo è conseguenza del fatto che ogni funzione calcolabile può essere espressa in un formalismo che
contiene somma, prodotto, costanti, proiezioni ed è chiuso per composizione e minimizzazione.

Mi serve anche l'uguaglianza nel mio formalismo.

\begin{thm}
    Ogni insieme ricorsivamente enumerabile è aritmetico.
\end{thm}
\begin{proof}
    Se $A$ è r.e. esiste $f$ parziale calcolabile tale che $A = \dom(f)$. Sia $\psi_{f}(x,y)$ una
    descrizione aritmetica di $f$. Allora:
    \begin{equation*}
        n \in A \iff \StdNat \models \exists y, \psi(\bar{n},y)
    \end{equation*}
    e dunque $\psi_{A}(x) = \exists y, \psi(x,y)$ è una descrizione aritmetica di $A$.
\end{proof}

\begin{thm}
    L'insieme delle formule aritmetiche vere non è ricorsivamente enumerabile.
\end{thm}
\begin{proof}
    Sia $\set{\psi_{n}}_{n \in \Nat}$ una enumerazione effettiva delle formule aritmetiche in una
    variabile. Consideriamo l'insieme $A$ così definito
    \begin{equation*}
        n \in A \iff \StdNat \models \lnot \psi_{n}(\bar{n})
    \end{equation*}
    Se la verità aritmetica fosse semidecidibile allora $A$ sarebbe r.e. Dunque $A$ dovrebbe essere
    aritmetico e dovrebbe esistere una formula $\psi_{a}$ per cui
    \begin{equation*}
        n \in A \iff \StdNat \models \lnot \psi_{a}(\bar{n})
    \end{equation*}
    Ma per $n=a$ otteniamo una contraddizione.
\end{proof}

\begin{thm}
    Ogni sistema formale aritmetico, se consistente, è incompleto (i.e. esistono formule
    aritmetiche valide ma non dimostrabili).
\end{thm}

Un sistema formale è definito da un insieme ricorsivo di assiomi e un insieme di
regole di inferenza che permettono di dedurre nuovi teoremi a partire dagli
assiomi in un numero finito di applicazioni.

Pertanto le formule aritmetiche dimostrabili costituiscono un insieme
ricorsivamente enumerabile.

Poichè le formule vere non sono r.e. esistono necessariamente delle formule
vere ma non dimostrabili.

È difficile descrivere in maniera categorica una teoria. Esistono poche teorie per cui non sia
possibile cambiare il modello e ribaltarla completamente.

\section{La gerarchia aritmetica}

L'uguaglianza tra numeri naturali è decidibile. Questo non è banale, ad esempio per i numeri reali
non è nemmeno semidecidibile l'uguaglianza; per essi è semidecidibile la disuguaglianza.

Data una formula aritmetica possiamo spostare tutti quantificatori all'inizio della formula. La mia
formula acquista quindi la forma $Q_{1} x_{1} Q_{2} x_{2} Q_{3}x_{3} \dotsc Q_{n}x_{n}P$, con $P$ senza
quantificatori e con $P$ decidibile. Se abbiamo dei quantificatori uniformi che si susseguono possiamo
collassarli. Ad esempio $\forall x_{1} \forall x_{2} \dotsc$ diventa $\forall <x_{1},x_{2}> \dotsc$. Nella
formula si parlerà poi di proiezioni della codifica al posto delle singole istanze delle variabili
collassate. Ad esempio $\textit{fst}(x)$ invece che $x_{1}$, $\pi_{n}(x)$ invece che $x_{n}$.

Nel collasso c'è un cambio di notazione. Il $\forall$ diventa $\Pi_{n}$, con $n$ uguale al numero di
inversioni $\forall$ $\exists$. Lo stesso discorso vale per $\exists$ e $\Sigma_{n}$. Ad esempio,
$\forall \exists \forall$ è di classe $\Pi_{3}$, e $\exists \forall$ è di classe $\Sigma_{2}$. Esiste poi
$\Delta_{n} = \Pi_{n} \cap \Sigma_{n}$.

Quali sono le formule $\Sigma_{1}$? Quelle semidecidibili. Perché? Perché sono proiezione
esistenziale di precidati decidibili.

Se $A \in \Sigma_{n}$ allora $\comp{A} \in \Pi_{n}$. $\Pi_{1}$ corrisponde alla classe degli insiemi
co-r.e. Come conseguenza ho che $\Delta_{1}$ è la classe degli insiemi ricorsivi.

La gerarchia degli insiemi è interessante perché dà una classificazione degli insiemi per
complessità computazionale.

Il ragionamento vale per tutti i tipi di proprietà, non solo quelle estensionali. In generale si
vuole dare una descrizione aritmetica di un programma. Da questa poi sarà possibile fare dei
ragionamenti per capire ``quanto sia difficile''. Ad esempio, sia $P(i) = ``W_{i} \text{ è
finito}$''. La sua
descrizione aritmetica sarebbe $\exists x \forall y \forall t \forall m, y \geq x \implies \lnot
T(i,y,m,t)$. Ho che $P(i) \in \Sigma_{2}$. Mi aspetto che non sia ne r.e. ne co-r.e., e in effetti non
lo è. Va comunque ricordato che l'appartenenza ad una classe non è mutalmente esclusiva. Potrei
avere proprietà $\Sigma_{2}$ che appartengono a $\Sigma_{1}$. Questo di solito accade quando i
quantificatori sono bound, che è la stessa situazione che avremmo se non esistessero.

In $\Delta_{n}$ stanno insiemi descrivibili sia con una formula in $\Pi_{n}$ che con una formula in
$\Sigma_{n}$.
