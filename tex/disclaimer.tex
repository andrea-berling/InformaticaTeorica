%% tex/disclaimer.tex
%% Copyright 2019 Andrea Berlingieri
%
% This work may be distributed and/or modified under the
% conditions of the LaTeX Project Public License, either version 1.3
% of this license or (at your option) any later version.
% The latest version of this license is in
%   http://www.latex-project.org/lppl.txt
% and version 1.3 or later is part of all distributions of LaTeX
% version 2005/12/01 or later.
%
% This work has the LPPL maintenance status `maintained'.
%
% The Current Maintainer of this work is Andrea Berlingieri.
%
% This work consists of all files listed in manifest.txt
\thispagestyle{plain}
\begin{center}
    \Large
    \textbf{Disclaimer}
\end{center}

Questo documento e' stato creato sulla base dalle lezioni del professore Andrea Asperti di
Informatica Teorica tenute nell'Anno Accademico 2018/2019 a Bologna. Il contenuto deriva in parte
dalle lezioni frontali e in parte direttamente dal materiale di supporto del corso, che comprende
dei lucidi ed una dispensa per la parte di Calcolabilita'. In ogni caso il contenuto e' stato
soggetto di una interpretazione personale del sottoscritto per quanto riguarda la forma, l'ordine e
alcuni dettagli degli argomenti presentati nel corso.

Di conseguenza, sebbene il professor Asperti sia a conoscenza dell'esistenza di questo documento e
abbia dato il suo consenso alla pubblicazione, non si tratta di materiale ufficiale del corso. Puo'
essere utilizzato come materiale di supporto allo studio della materia e spero che risulti utile a
questo scopo. Cio' detto il documento non e' stato certificato da nessuno e non posso garantire che
sia esente da refusi o errori logici, i quali possono essere stati frutto di distrazione personale
durante le lezioni, di una interpretazione erronea dei contenuti trattati, o di altri fattori. In
ogni caso mi assumo la responsabilita' di qualsiasi errore presente nel documento. Vi prego di non
disturbare il professor Asperti al riguardo ma piuttosto di rivolgervi a me.

Ribadisco nuovamente che non c'e' alcuna garanzia di correttezza di questo documento, e che potete
farne uso a vostro rischio e pericolo. Se un refuso nel documento vi portasse, ad esempio, a
sbagliare una risposta all'esame sappiate che non c'e' nessuno a cui potreste fare ricorso per
questo tipo di incidente. Vi assumete il rischio di sbagliare se vi basate interamente e ciecamente
su questo documento.

Cio' detto ci tengo a dire che nel creare questo documento ci ho messo tutto il mio impegno per
ottenere un risultato che fosse il piu' chiaro, corretto e completo possibile per i miei sforzi.
Spero che questo si rifletta nel risultato finale e che lo apprezziate. Ogni correzione,
suggerimento, o partecipazione costruttiva al lavoro e' bene accetta e sara' dovutamente
riconosciuta quando andro' a stendere la lista delle persone che hanno contribuito a creare questo
documento.

Spero infine, ancora una volta, che questo materiale vi aiuti nello studio della materia e che
risulti il piu' utile possibile. Buono studio.
